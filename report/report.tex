\documentclass[a4paper, 12pt]{article}
\usepackage[swedish]{babel}
\usepackage[utf8]{inputenc}
\usepackage{verbatim}
\usepackage{fancyhdr}
\usepackage{graphicx}
\usepackage{parskip}
\usepackage{minitoc}

% Include pdf with multiple pages ex \includepdf[pages=-, nup=2x2]{filename.pdf}
\usepackage[final]{pdfpages}
% Place figures where they should be
\usepackage{float}

% Float for text
\floatstyle{ruled}
\newfloat{xml}{H}{lop}
\floatname{xml}{XML}

% vars
\def\title{Webbtjänster}
\def\preTitle{Laboration 3}
\def\kurs{Applikationsprogrammering i Java, HT-08}

\def\namn{Anton Johansson}
\def\mail{dit06ajn@cs.umu.se}
\def\pathtocode{$\sim$dit06ajn/edu/apjava/lab3}

\def\handledareEtt{Johan Eliasson, johane@cs.umu.se}
\def\handledareTva{Tor Sterner-Johansson, tors@cs.umu.se}
\def\handledareTre{Daniel Henriksson, danielh@cs.umu.se}
\def\handledareFyra{Johan Granberg, johang@cs.umu.se}

\def\inst{datavetenskap}
\def\dokumentTyp{Laborationsrapport}

\begin{document}
\begin{titlepage}
  \thispagestyle{empty}
  \begin{small}
    \begin{tabular}{@{}p{\textwidth}@{}}
      UMEÅ UNIVERSITET \hfill \today \\
      Institutionen för \inst \\
      \dokumentTyp \\
    \end{tabular}
  \end{small}
  \vspace{10mm}
  \begin{center}
    \LARGE{\preTitle} \\
    \huge{\textbf{\kurs}} \\
    \vspace{10mm}
    \LARGE{\title} \\
    \vspace{15mm}
    \begin{large}
        \namn, \mail \\
        \texttt{\pathtocode}
    \end{large}
    \vfill
    \large{\textbf{Handledare}}\\
    \mbox{\large{\handledareTre}}
    \mbox{\large{\handledareTva}}
    \mbox{\large{\handledareEtt}}
    \mbox{\large{\handledareFyra}}
  \end{center}
\end{titlepage}

\newpage
\mbox{}
\vspace{70mm}
\begin{center}
% Dedication goes here
\end{center}
\thispagestyle{empty}
\newpage

\pagestyle{fancy}
\rhead{\today}
\lhead{\namn, \mail}
\chead{}
\lfoot{}
\cfoot{}
\rfoot{}

\cleardoublepage
\newpage
\dosecttoc 
\tableofcontents
\cleardoublepage

\rfoot{\thepage}
\pagenumbering{arabic}

\section{Problemspecifikation}\label{Problemspecifikation}
% Beskriv med egna ord vad uppgiften gick ut på. Är det någonting som
% varit oklart och ni gjort egna tolkningar så beskriv dessa.
Laborationen gick ut på att implementera en webbtjänst för att skicka
och ta emot information över nätet, så kallad
interprocesskommunikation, från engelskans \textit{inter-process
  communication}, förkortat \textit{IPC}. Givet var en XML-fil, där
webbtjänsten var beskriven enligt \textit{WSDL}, \textit{Web Services
  Description Language}. Denna fil specificerar hur meddelanden ska
vara utformade när de skickas och hur svaren kommer att se
ut. Kommunikationen är även den XML-baserad och sköts via protokollet
HTTP.

Webbtjänsten skulle ha funktionalitet som hanterar att från ett anrop
spara en poängregistrering, bestående av namn, datum och
poäng. Tjänsten ska även via anrop kunna returnera alla
poäng\-registreringar som är sparade. Till detta skulle även en klient
implementeras. Klienten skulle kunna skicka meddelande för att spara
och hämta poäng till och från tjänsten.

För denna implementering var det givet en utvecklingsmiljö bestående
av en webbserver och ett påbörjat projekt där vissa klasser var
givna/genererade. Utvecklingsmiljön finns på sidan:\\
\begin{footnotesize}
\verb!http://www.cs.umu.se/kurser/5DV085/HT08/assignments/environment/!
\end{footnotesize}

Problemspecifikation finns i original på sidan:\\
\begin{footnotesize}
\verb!http://www.cs.umu.se/kurser/5DV085/HT08/assignments/3/!
\end{footnotesize}

\section{Användarhandledning}\label{Anvandarhandledning}
% Förklara var programmet och källkoden ligger samt hur man kompilerar,
% startar och använder det. Förklara även översiktligt vad som händer
% när man använder de olika kommandona. Det räcker alltså inte att
% skriva "man skriver 'ant' för att kompilera", utan det måste även ingå
% en liten förklaring om vad som egentligen händer när man kör ant och
% varför det fungerar. Använd Internet eller litteratur för att själva
% ta reda på den information ni tycker känns relevant, dels för
% rapportens skull och dels för er egen. Kom ihåg att skriva tydliga
% (vetenskapliga) referenser!

Programmet ligger i katalogen:\\
\texttt{\pathtocode}

Källkoden ligger i underkatalogen \verb!src!.

För att köra detta program krävs en webb-servern med bifogade paket
som finns i
utvecklingsmiljön\footnote{http://www.cs.umu.se/kurser/5DV085/HT08/assignments/environment/}.
I denna miljö följer även ett antal skript för att starta,
uppdatera och stänga av webb-servern.

Följande kommandon förutsätter att programmet \textit{Apache
  Ant}\footnote{http://ant.apache.org/} är installerat och att mappen
\verb!apache-tomcat-6.0.16!  från utvecklingsmiljön är uppackad och
ligger i katalogen över det där programmet ligger, det vill säga:
\verb!~dit06ajn/apjava/apache-tomcat-6.0.16!.

Verktyget \textit{Ant} är ett byggverktyg som använder sig av
specifikationen lagrad i en XML-fil, oftast \textit{build.xml}, för
automatisera alla nödvändiga operationen vid kompilering av ett
projekt. Detta kan innefatta all typ av filhantering, det vill säga
kopiering, borttagning och flyttning, men även själva kompileringen av
projektet. Verktyget kan ses som ett specialanpassat skript för att
kompilera hela projekt.

% TODO: flytta apache-tomcat rätt
Programmet kompileras med kommandot:\\
\begin{footnotesize}
\verb!salt:~/edu/apjava/lab3> ant!
\end{footnotesize}

Det som händer vid anrop av kommandot ovan är att \textit{Ant} läser
av filen \textit{build.xml} och letar efter standardkommandona att
köra. I det här fallet är det operationerna som är definierade i
XML-elementet \verb!<target />! med attributet \verb!name="dist"!. Se
bildaga~\ref{build.xml} för mer information om vad som händer. Oftast
har taggarna i \textit{build.xml} relativt självförklarande namn, de
motsvarar i många fall direkta kommandon som går att köra i en
terminal.

För att starta webb-servern och sätta igång webbtjänsten körs skripten:\\
\begin{footnotesize}
\verb!salt:~/edu/apjava/lab3> ./deploy-linux.sh !\\
\verb!salt:~/edu/apjava/lab3> ./start-linux.sh !
\end{footnotesize}

När detta är gjort finns webbtjänsten redo att användas lokalt via:\\
\begin{footnotesize}
\verb!http://localhost:37080/axis2/services/HighScoreService!
\end{footnotesize}

\subsection{Spara poäng}

För att spara en eller flera poäng till tjänsten används lättast
skriptet\linebreak \texttt{store-linux.sh} som i sin tur anropar den
implementerade Java-klienten \textit{StoreClient}, se källkod bilaga
\ref{StoreClient.java}. Till detta skript skickas parametrar med som
anger \textit{namn}, \textit{datum} och \textit{poäng}. För att skicka
in flera poäng\-registreringar till tjänsten samtidigt kan flera
parametrar på rad tas emot. Exempel:

\begin{footnotesize}
\verb!salt:~/edu/apjava/lab3> ./store-linux.sh Anton tors 100 Olle fre 123!
\end{footnotesize}

Kommandot ovan kommer att spara en registrering på namn \verb!Anton!,
datum \verb!tors! med poäng \verb!100! och en registrering på namn
\verb!Olle!, datum \verb!fre! med poäng \verb!123!. Det finns inga
restriktioner på vad som kan sparas i varje fält förutom att
mellanslag inte kan förekomma när man anropar klienten via detta
skript.

\subsection{Hämta poäng}

För att hämta poäng som finns sparade i webbtjänsten körs lättast
skriptet \texttt{retrieve-linux.sh} som anropar Java-klienten
\textit{RetrieveClient}, se källkod i
bilaga~\ref{RetrieveClient.java}. Exempel:

\begin{footnotesize}
\verb!salt:~/edu/apjava/lab3> ./retrieve-linux.sh !
\end{footnotesize}

Utöver eventuell debug-utskrift listas från kommandot ovan ut alla
poäng\-registreringar som finns sparade i tjänsten:

\begin{footnotesize}
\begin{verbatim}
Retrieved 2 entries.
Name:  Anton
Date:  tors
Score: 100

Name:  Olle
Date:  fre
Score: 123
\end{verbatim}
\end{footnotesize}

\subsection{Inställningar}
För att ändra adressen till webbtjänsten som klienterna,
\textit{StoreClient} och \textit{RetrieveClient}, ska anropa sätts en
\textit{System property}, \verb!service.url! till önskad URL.

För att slippa debug-utskrifter sätts, \textit{System property}, 
\verb!debug.messages! till allt förutom \textit{true}. Som standard
sätts denna property till \textit{true} inuti skripten
\verb!store-linux.sh! och \verb!retrieve-linux.sh!.

\subsection{Webbgränssnitt}
Ett webbgränssnitt finns implementerat och kan testas via sidan:

\begin{footnotesize}
\verb!http://salt.cs.umu.se:37080/dit06ajn/!
\end{footnotesize}

Detta gränssnitt och tjänsten den använder kommer att finnas
tillgänglig någon vecka framåt från och med 9 december 2008. Källkod
finns i bilaga \ref{index.jsp}.

\section{Systembeskrivning}\label{Systembeskrivning}
% Beskriv översiktligt hur programmet är uppbyggt och hur det löser
% problemet.

Systemet består i huvudsak av två delar, en klient och en
server. Dessa två används oberoende av varandra men använder sig vid
kompilering av två gemensamma klasser, \textit{Entry} och
\textit{XMLUtil}. Det som gör att klienten och servern kan prata med
varandra är att de använder sig av samma specifikation för meddelanden
som skickas och tas emot, se avsnitt~\ref{sec:highscoreservice.wsdl}.

Nedan avsnitt går igenom systemets olika klasser, även andra klasser
är aktiva i webbtjänsten, dessa är dock genererade utifrån
\textit{WSDL}-filen och kommer inte att beskrivas i denna rapport.

\subsection{Entry}
Klassen \textit{Entry} fanns bifogad i utvecklingsmiljön. Klassen
representerar en poängregistrering i webbtjänsten, namn, datum och
poäng kan sparas undan och hämtas ut.

Källkod finns i bilaga~\ref{Entry.java}.

\subsection{XMLUtil}
Klassen \textit{XMLUtil} används av både servern och klienterna för
att tolka XML-element som innehåller ett eller flera
\textit{score}-element av typen \textit{EntryType} beskrivet i
XML-schemat från \textit{WSDL}-filen. Klassen har även en metod som
kan skapa en eller flera instanser av klassen \textit{Entry} från
givet XML-element beskrivet ovan.

Källkod finns i bilaga~\ref{XMLUtil.java}.

\subsection{HighScoreServiceClient}
Klassen \textit{HighScoreServiceClient} implementerar två metoder
\textit{store} och \textit{retrive}. Klassens konstruktor tar emot en
URL som anger adressen webbtjänsten som ska användas.

Metoden \textit{store} tar emot en vektor med instanser av klassen
\textit{Entry}, formaterar dessa till ett XML-element av typen
\textit{StoreRequestType}, enligt WSDL-fil, och skickar dessa till den
aktuella webbtjänsten.

Metoden \textit{retrieve} hämtar alla poäng\-registreringar sparade i
den aktuella webbtjänsten och returnerar dessa i en vektor
innehållandes instanser av klassen \textit{Entry}.

Källkod finns i bilaga~\ref{HighScoreServiceClient.java}.

\subsection{RetrieveClient}
Klassen \textit{RetrieveClient} är till för att via en terminal på ett
enkelt sätt kunna anropa klassen \textit{HighScoreServiceClient}s
metod \textit{retrieve}. Vid start av \textit{RetrieveClient} kollas
\textit{System property}, \verb!service.url!, är denna satt ändras
adressen till tjänsten till det nya värdet. Standardvärde är:

\begin{footnotesize}
\begin{verbatim}
http://localhost:37080/axis2/services/HighScoreService
\end{verbatim}
\end{footnotesize}

Finns det poäng\-registreringar sparade i tjänsten skrivs dessa ut till
\textit{standard output}, det vill säga terminalen
\textit{RetrieveClient} anropades från.

Källkod finns i bilaga~\ref{RetrieveClient.java}.

\subsection{StoreClient}
Klassen \textit{StoreClient} är till för att via en terminal på ett
enkelt sätt kunna anropa klassen \textit{HighScoreServiceClient}s
metod \textit{store}. Vid start av \textit{RetrieveClient} kollas
\textit{System property}, \verb!service.url!, är denna satt ändras
adressen till tjänsten till det nya värdet. Standardvärde är:

\begin{footnotesize}
\begin{verbatim}
http://localhost:37080/axis2/services/HighScoreService
\end{verbatim}
\end{footnotesize}

Vid anrop av \textit{StoreClient} läses eventuella parametrar av och
läggs gruppvis, om tre, in som namn, datum och poäng i nya instanser av
klassen \textit{Entry}. Dessa skickas sedan med till metoden
\textit{store} i klassen \textit{HighScoreServiceClient}. Dessa
poäng\-registreringar ska sparas i webbtjänsten.

Källkod finns i bilaga~\ref{StoreClient.java}.

\subsection{HighScoreServiceData}
Klassen \textit{HighScoreServiceData} sparar undan all information i
webbtjänsten. Eftersom varje anrop till webbtjänsten genererar en ny
instans av klassen \textit{HighScoreServiceSkeleton} används
\textit{HighScoreServiceData} som en \textit{Singleton}, se boken
\textit{Design Patterns}~\cite{gamma}. Detta innebär att så länge
webbtjänsten är igång kommer en och enbart en instans att existera av
denna klass. En statisk metod för att hämta denna instans finns
implementerad, utöver det finns metoder för att spara undan och hämta
poäng\-registreringar av typen \textit{Entry}.

Källkod finns i bilaga~\ref{HighScoreServiceData.java}.

\subsection{HighScoreServiceSkeleton}
Klassen \textit{HighScoreServiceSkeleton} hanterar slutändan av de
faktiska till webbtjänsten. Klassens metoder \textit{store} och
\textit{retrieve} sköter logiken för de två olika anropen till
webbtjänsten.

Metoden \textit{store} tar emot ett XML-element som kan innehålla ett
eller flera \textit{score}-element av typen \textit{EntryType}. Dessa
tolkas av klassen \textit{XMLUtil} och sparas undan som instanser av
typen \textit{Entry} i klassen \textit{HighScoreServiceData}. Ett tomt
XML-element av typen \textit{StoreResponse} returneras.

Metoden \textit{retrieve} hämtar ut alla poäng\-registreringar ur
instansen av \textit{HighScoreServiceData} lägger in dem i ett
XML-element med hjälp av en metod i klassen \textit{XMLUtil}. Detta
element returneras från metoden.

Källkod finns i bilaga~\ref{HighScoreServiceSkeleton.java}.

\subsection{Filen highscoreservice.wsdl}\label{sec:highscoreservice.wsdl}
Filen \textit{highscoreservice.wsdl} är definitionen av själva
webbtjänsten. Här beskrivs de metoder och meddelanden som tjänsten
arbetar med. Två metoder \textit{store} och \textit{retrieve} finns
beskrivna, dess indata och utdata är specificerade i ett
XML-schema.

Operationen \textit{store} har indata enligt
XML~\ref{StoreRequest}. Utdata är ett XML-element
\verb!<StoreResponse/>!.

Operationen \textit{retrieve} har ett XML-element
\verb!<RetrieveRequest/>! som indata. Utdata från operationen följer
utseendet på XML~\ref{RetrieveResponse}.

\begin{xml}
\begin{verbatim}
<StoreRequest>
  <score>
    <name></name>
    <date></date>
    <score></score>
  </score>
  <!-- fler scores-element -->
</StoreRequest>
\end{verbatim}
  \caption{StoreRequest}\label{StoreRequest}
\end{xml}


\begin{xml}
\begin{verbatim}
<RetrieveResponse>
  <score>
    <name></name>
    <date></date>
    <score></score>
  </score>
  <!-- fler scores-element -->
</RetrieveResponse>
\end{verbatim}
  \caption{RetrieveResponse}\label{RetrieveResponse}
\end{xml}

Filens \textit{highscoreservice.wsdl} innehåll finns i
bilaga~\ref{highscoreservice.wsdl}.

\section{Begränsningar}\label{Begransningar}
% Vilka problem och begränsningar har din lösning av uppgiften? Hur
% skulle de kunna rättas till?
Varken servern eller klienten säkerhetställer korrekta meddelanden in
och ut genom att validera mot schemat som finns beskrivet i
\textit{WSDL}-filen. Detta skulle kunna lösas genom att extrahera
schemat från denna fil och validera XML-elementen som skickas och tas
emot mot denna information.

Eftersom alla meddelanden skickas som text-strängar kommer namn, datum
och poäng kunna innehålla vad som helst. Detta borde gå att lösa genom
att införa restriktioner i XML-schemat som finns i WSDL-filen. Dock
blir detta en ändring av laborationsspecifikationen vilket innebär att
det kanske borde ses som en annan lösning och inte som en begränsning.

Tjänsten som är implementerad sparar inte undan informationen om
poäng\-registreringar till disk eller till någon typ av databas. Detta
medför att varje gång servern startas om så kommer all information att
gå förlorad. En lösning till detta är att spara ner informationen till
antingen XML-filer/fil eller att använda en databas.

Sättet att hantera undantag i klassen
\textit{HighScoreServiceSkeleton} skickar med alla typer av undantag
med all information om undantaget till klienten. Detta kan innebära
vissa säkerhetsrisker och bör ses över.

\section{Reflektioner}\label{Reflektioner}
% Reflektioner - Var det något som var speciellt krångligt? Vilka
% problem uppstod och hur löste ni dem? Vilka verktyg använde ni? Hur
% upplevde ni de verktygen? + Allmänna synpunkter. Om ni har upplevt
% problem på grund av olika miljöer (i termer av operativsystem och
% liknande) så kan det även vara intressant att nämna det, samt motivera
% ert val av miljö.

Svårigheterna med denna laboration bestod i att få ihop
utvecklingsmiljön och förstå hur de olika skripten och klasserna
hängde ihop och kommunicerade.

Jag använde mig till största del av operativsystemet \textit{OSX
  Leopard} och textredigeraren \textit{emacs} för att utveckla detta
program. Detta på grund av vana och eftersom det innebär att det utan
modifikationer går flytta projektet till skolans datorer och köra dem
där med samma skript. Dock flyttade jag och ändrade i de givna
skripten för att använda mig av den version av \textit{Java} och
\textit{Ant} som redan är installerade på respektive datorer.

Detta system hade ändras något och användas som ett komplement till
ett spel där man vill spara en global rekordlista tillgänglig för alla
som har Internetuppkoppling.

\section{Testkörningar}\label{Testkorningar}
% Noggranna testkörningar där man ser att programmet fungerar som det
% ska.
Följande avsnitt beskriver några olika scenarion vid användande av
webbtjänsten. Alla kommandon som följer förutsätter att en webbserver
är igång på adress:
\begin{footnotesize}
\verb!http://localhost:37080/axis2/services/HighScoreService!
\end{footnotesize}

Förutsatt är även att \textit{System property}, \verb!service.url!
inte är ändrat.

Programmet \textit{Apache
  TCPMon}\footnote{http://ws.apache.org/commons/tcpmon/} är använt för
att ''tjuvlyssna'' på anrop till och från webb-servern.

\subsection{Spara poäng}\label{sec:spara}

För att spara en poängregistrering körs:

\begin{footnotesize}
\verb!salt:~/edu/apjava/lab3> ./store-linux.sh anton idag 101!
\end{footnotesize}

Ovan kommando genererar ett anrop till webbtjänsten. Den del av
anropet som denna laboration har handlat om ses i
XML~\ref{xml:store1}.

\begin{xml}
  \begin{footnotesize}
\begin{verbatim}
<store:StoreRequest
 xmlns:store="http://nemi.cs.umu.se:8080/axis2/services/HighScoreService">
   <store:score>
      <store:name>anton</store:name>
      <store:date>idag</store:date>
      <store:score>101</store:score>
   </store:score>
</store:StoreRequest>
\end{verbatim}
  \end{footnotesize}
  \caption{StoreRequest, en poängregistrering}\label{xml:store1}
\end{xml}

Webbservern loggar meddelanden från tjänsten i filen:

\begin{footnotesize}
\verb!~dit06ajn/apjava/apache-tomcat-6.0.16/logs/catalina.out!
\end{footnotesize}

Nedan textrader är resultatet i filen \verb!catalina.out! efter
anropet i XML~\ref{xml:store1}:

\begin{footnotesize}
\begin{verbatim}
[INFO] New instance of HighScoreServiceSkeleton created.
[INFO] New instance fo HighScoreServiceData created.
[INFO] New entry stored: 101 (anton @ idag).
[INFO] Added entry: 101 (anton @ idag)
[INFO] All entires in HighScoreServiceData:
101 (anton @ idag)
\end{verbatim}
\end{footnotesize}

För att spara två poäng\-registreringar samtidigt körs: 

\begin{footnotesize}
\verb!salt:~/edu/apjava/lab3> ./store-linux.sh Pelle tors 12 Lena fre 142!
\end{footnotesize}

Detta resulterar i anropet som syns i XML~\ref{xml:store2}. Nedan
textrader är förändringen i \verb!catalina.out!:

\begin{footnotesize}
\begin{verbatim}
[INFO] New instance of HighScoreServiceSkeleton created.
[INFO] New entry stored: 12 (Pelle @ tors).
[INFO] Added entry: 12 (Pelle @ tors)
[INFO] New entry stored: 142 (Lena @ fre).
[INFO] Added entry: 142 (Lena @ fre)
[INFO] All entires in HighScoreServiceData:
101 (anton @ idag)
12 (Pelle @ tors)
142 (Lena @ fre)
\end{verbatim}
\end{footnotesize}

Alla \textit{StoreRequest}-anrop genererar ett svar som ses i
XML~\ref{xml:storeresponse}.

\begin{xml}
  \begin{footnotesize}
\begin{verbatim}
<store:StoreRequest
 xmlns:store="http://nemi.cs.umu.se:8080/axis2/services/HighScoreService">
   <store:score>
      <store:name>Pelle</store:name>
      <store:date>tors</store:date>
      <store:score>12</store:score>
   </store:score>
   <store:score>
      <store:name>Lena</store:name>
      <store:date>fre</store:date>
      <store:score>142</store:score>
   </store:score>
</store:StoreRequest>
\end{verbatim}
  \end{footnotesize}
  \caption{StoreRequest, två poängregistreringar samtidigt}\label{xml:store2}
\end{xml}

\begin{xml}
  \begin{footnotesize}
\begin{verbatim}
<store:StoreResponse
 xmlns:store="http://nemi.cs.umu.se:8080/axis2/services/HighScoreService" />
\end{verbatim}
  \end{footnotesize}
  \caption{StoreResponse}\label{xml:storeresponse}
\end{xml}

\subsection{Hämta poäng}
Efter poäng\-registreringarna i sektion \ref{sec:spara} är gjorda
kommer webbtjänsten att innehålla tre poster. Dessa kan hämtas med
kommandot:

\begin{footnotesize}
\verb!salt:~/edu/apjava/lab3> ./retrieve-linux.sh!
\end{footnotesize}

Är \textit{System property}, \verb!debug.messages! inte satt till
\textit{true}, kommer ovan kommando resultera i följande utskrift:

\begin{footnotesize}
\begin{verbatim}
Retrieved 3 entries.
Name:  anton
Date:  idag
Score: 101

Name:  Pelle
Date:  tors
Score: 12

Name:  Lena
Date:  fre
Score: 142
\end{verbatim}
\end{footnotesize}

Anropet som skickas till servern består av meddelandet i
XML~\ref{xml:retrieverequest}. Det som fås tillbaka från webbtjänsten
visas i XML~\ref{xml:retrieveresponse}.

\begin{xml}
  \begin{footnotesize}
\begin{verbatim}
<retrieve:RetrieveRequest
 xmlns:retrieve="http://nemi.cs.umu.se:8080/axis2/services/HighScoreService" />
\end{verbatim}
  \end{footnotesize}
  \caption{RetrieveRequest}\label{xml:retrieverequest}
\end{xml}

\begin{xml}
  \begin{footnotesize}
\begin{verbatim}
<retrieve:RetrieveResponse
 xmlns:retrieve="http://nemi.cs.umu.se:8080/axis2/services/HighScoreService">
   <retrieve:score>
      <retrieve:name>anton</retrieve:name>
      <retrieve:date>idag</retrieve:date>
      <retrieve:score>101</retrieve:score>
   </retrieve:score>
   <retrieve:score>
      <retrieve:name>Pelle</retrieve:name>
      <retrieve:date>tors</retrieve:date>
      <retrieve:score>12</retrieve:score>
   </retrieve:score>
   <retrieve:score>
      <retrieve:name>Lena</retrieve:name>
      <retrieve:date>fre</retrieve:date>
      <retrieve:score>142</retrieve:score>
   </retrieve:score>
</retrieve:RetrieveResponse>
\end{verbatim}
  \end{footnotesize}
  \caption{RetrieveResponse}\label{xml:retrieveresponse}
\end{xml}

\section{Diskussion}\label{Diskussion}
% Diskutera om laborationen samt allmänt kring Web services och om hur
% och när det är användbart (och inte användbart). Saker som kan vara
% trevliga att ta upp är interoperabilitet, lite om prestanda, koncepten
% lös koppling och så vidare. Den här sektionen ska vara en betydande
% del av rapporten. Det är upp till er själva att ta reda på den
% information ni behöver, även om föreläsningsmaterialet kan vara
% väldigt användbart. Kom även här ihåg att referera till era källor
% (även om det är från föreläsningsmaterialet).

Eftersom denna webbtjänst enbart använder sig av anrop innehållandes
vanlig text kommer servern kunna användas från alla system som kan
hantera text. Anropen sköts via protokollet HTTP, alltså räcker det
med en webbläsare på klientsidan för att kunna använda tjänsten.

Denna typ av tjänster kan vara väldigt användbara för möjligheten att
erbjuda information till andra över hela jordklotet och oberoende av
vilket system de använder sig av. Systemoberoendet gör att denna
lösning lämpar sig väl i situationer där information finns som ska
distribueras på framtidssäker sätt och där klienternas system kan
förrändras.

Nackdelen med att ha en text/xml-baserad kommunikation är att mycket
av den faktiska datan som skickas över nätet kommer att innehålla
meta-information om själva datan som skickas. Det är lätt att se i
denna laboration där tre korta strängar skickas inbäddade i långa
start- och slut-taggar. För mer mängd information som ska färdas långt
över Internet kan detta sätt att skicka data sluka väldigt mycket
bandbredd och därför också gå långsamt.

Med så kallad \textit{lös koppling}, \textit{Loose Coupling} se pdf
\cite{web:po}, erbjuds möjligheter för system att ha minimal kännedom
om varandras interna implementationer. Webbtjänster hittas dynamiskt
och kommunicerar initialt genom att skicka gränssnittsbeskrivingar,
såsom \textit{WSDL}, mellan varandra. Möjligheten att på detta
dynamiska sätt erbjuda en tjänst från flera distribuerade servar
innebär att systemet blir väldigt skalbart och felsäkert. I ett
exempel från föreläsning~\cite{lec:po} drogs en liknelse till hur
telefonkatalogen fungerar, företag anmäler sina tjänster till ett
register (telefonkatalogen) och klienter/personer letar i registret
efter önskad tjänst (telefonnummer/IP-nummer) för att sedan sköta
resten av kommunikationenen direkt med önskad tjänst. Denna typ av
lösning kallas även \textit{Publish, Bind, Find}.

\bibliographystyle{alpha}
\bibliography{books.bib}

\newpage
\appendix
\pagenumbering{arabic}
\section{Bilagor}\label{Bilagor}
% Källkoden ska finnas tillgänglig i er hemkatalog
% ~/edu/apjava/lab1/. Bifoga även utskriven källkod.
Härefter följer utskrifter från källkoden och andra filer som hör till
denna laboration.

\newpage
\subsection{Entry.java}\label{Entry.java}
\begin{footnotesize}
  \verbatiminput{../src/se/umu/cs/edu/jap/highscoreservice/Entry.java}
\end{footnotesize}

\newpage
\subsection{HighScoreServiceClient.java}\label{HighScoreServiceClient.java}
\begin{footnotesize}
  \verbatiminput{../src/se/umu/cs/edu/jap/highscoreservice/HighScoreServiceClient.java}
\end{footnotesize}

\newpage
\subsection{RetrieveClient.java}\label{RetrieveClient.java}
\begin{footnotesize}
  \verbatiminput{../src/se/umu/cs/edu/jap/highscoreservice/RetrieveClient.java}
\end{footnotesize}

\newpage
\subsection{StoreClient.java}\label{StoreClient.java}
\begin{footnotesize}
  \verbatiminput{../src/se/umu/cs/edu/jap/highscoreservice/StoreClient.java}
\end{footnotesize}

\newpage
\subsection{HighScoreServiceData.java}\label{HighScoreServiceData.java}
\begin{footnotesize}
  \verbatiminput{../src/se/umu/cs/edu/jap/highscoreservice/stubs/HighScoreServiceData.java}
\end{footnotesize}

\newpage
\subsection{HighScoreServiceSkeleton.java}\label{HighScoreServiceSkeleton.java}
\begin{footnotesize}
  \verbatiminput{../src/se/umu/cs/edu/jap/highscoreservice/stubs/HighScoreServiceSkeleton.java}
\end{footnotesize}

\newpage
\subsection{XMLUtil.java}\label{XMLUtil.java}
\begin{footnotesize}
  \verbatiminput{../src/se/umu/cs/edu/jap/highscoreservice/util/XMLUtil.java}
\end{footnotesize}

\newpage
\subsection{store-linux.sh}\label{store-linux.sh}
\begin{footnotesize}
  \verbatiminput{../store-linux.sh}
\end{footnotesize}

\newpage
\subsection{retrieve-linux.sh}\label{etrieve-linux.sh}
\begin{footnotesize}
  \verbatiminput{../retrieve-linux.sh}
\end{footnotesize}

\newpage
\subsection{highscoreservice.wsdl}\label{highscoreservice.wsdl}
\begin{footnotesize}
  \verbatiminput{../wsdl/highscoreservice.wsdl}
\end{footnotesize}

\newpage
\subsection{build.xml}\label{build.xml}
\begin{footnotesize}
  \verbatiminput{../build.xml}
\end{footnotesize}

\newpage
\subsection{index.jsp}\label{index.jsp}
\begin{footnotesize}
  \verbatiminput{../../apache-tomcat-6.0.16/webapps/dit06ajn/index.jsp}
\end{footnotesize}

\end{document}