\documentclass[a4paper, 12pt]{article}
\usepackage[swedish]{babel}
\usepackage[utf8]{inputenc}
\usepackage{verbatim}
\usepackage{fancyhdr}
\usepackage{graphicx}
\usepackage{parskip}
% Include pdf with multiple pages ex \includepdf[pages=-, nup=2x2]{filename.pdf}
\usepackage[final]{pdfpages}
% Place figures where they should be
\usepackage{float}

% vars
\def\title{RSS}
\def\preTitle{Laboration 2}
\def\kurs{Applikationsprogrammering i Java, HT-08}

\def\namn{Anton Johansson}
\def\mail{dit06ajn@cs.umu.se}
 \def\pathtocode{$\sim$dit06ajn/edu/apjava/lab2}

\def\handledareEtt{Johan Eliasson, johane@cs.umu.se}
\def\handledareTva{Tor Sterner-Johansson, tors@cs.umu.se}
\def\handledareTre{Daniel Henriksson, danielh@cs.umu.se}

\def\inst{datavetenskap}
\def\dokumentTyp{Laborationsrapport}

\begin{document}
\begin{titlepage}
  \thispagestyle{empty}
  \begin{small}
    \begin{tabular}{@{}p{\textwidth}@{}}
      UMEÅ UNIVERSITET \hfill \today \\
      Institutionen för \inst \\
      \dokumentTyp \\
    \end{tabular}
  \end{small}
  \vspace{10mm}
  \begin{center}
    \LARGE{\preTitle} \\
    \huge{\textbf{\kurs}} \\
    \vspace{10mm}
    \LARGE{\title} \\
    \vspace{15mm}
    \begin{large}
        \namn, \mail \\
        \texttt{\pathtocode}
    \end{large}
    \vfill
    \large{\textbf{Handledare}}\\
    \mbox{\large{\handledareEtt}}
    \mbox{\large{\handledareTva}}
    \mbox{\large{\handledareTre}}
  \end{center}
\end{titlepage}

\pagestyle{fancy}
\rhead{\today}
\lhead{\namn, \mail}
\chead{}
\lfoot{}
\cfoot{}
\rfoot{}

\tableofcontents
\newpage

\rfoot{\thepage}
\pagenumbering{arabic}

\section{Problemspecifikation}\label{Problemspecifikation}
% Beskriv med egna ord vad uppgiften gick ut på. Är det någonting som
% varit oklart och ni gjort egna tolkningar så beskriv dessa.
Denna laboration gick ut på att skriva en RSS-läsare med grafiskt
användargränssnitt i programmeringsspråket Java. RSS står för Really
Simple Syndication och är ett filformat som används för att på ett
standardiserat sätt spara undan information i en lista. Oftast består
denna information av något typ av artikel.

En RSS-läsare fungerar ungefär som en mailklient, det vill säga
användare kan välja att prenumerera på valda RSS-flöden. Läsaren visar
upp artiklarna som finns i RSS-flödet, dessa kan läsas och markeras då
oftast som lästa på något sätt.

Exempel på RSS-flöde:\\
\verb!http://sydsvenskan.se/senastenytt/?context=senasteNyttRss!

Problemspecifikation finns i original på:\\
\verb!http://www.cs.umu.se/kurser/5DV085/HT08/labbar/lab2.html!

\section{Användarhandledning}\label{Anvandarhandledning}
% Förklara var programmet och källkoden ligger samt hur man startar,
% kompilerar och använder det.
Programmet ligger i katalogen:\\
\texttt{\pathtocode}

Från denna katalog kompileras programmet med kommandot:

\verb!salt:~/edu/apjava/lab2> ant!

De kompilerade filerna hamnar i underkatalogen \verb!bin!. Från denna
katalog körs programmet med kommandot

\verb!salt:~/edu/apjava/lab2/bin> java -cp .:../lib/jdom.jar JeedReader!

Alternativt kan programmet köras direkt från huvudmappen med
kommandot:

\verb!salt:~/edu/apjava/lab2> ant test-run!

När programmet startar läses föregående sparade flöden från katalogen
\verb!bin/config! in. Är det första gången programmet startar kommer
denna katalog att vara tom. Finns det flöden i katalogen kommer
gränssnittet att se ut ungefär som i figur \ref{fig:gui-out}.

\begin{figure}[H]
  \begin{center}
    \includegraphics[width=110mm]{images/gui-out.png}
    \caption{Användargränssnitt}
    \label{fig:gui-out}
  \end{center}
\end{figure}

Gränssnittet består en lista till vänster som innehåller flöden
användaren lagt till. Ovan till höger visas titeln för varje artikel i
valt flöde och under visas beskrivningen av vald artikel.

När en artikel är vald markeras denna som läst genom att texten
\verb!(Ny)! försvinner från dess titel i listan. Det går att lägga
till nya flöden via knappen \textit{Add} eller via menyn
\textit{Feeds/Add feed}.

För att uppdatera alla flöden används knappen \textit{Refresh} eller
menyvalet \textit{Feeds/Update~all~feeds}. För att enbart uppdatera
markerat flöde används menyvalet \textit{Feeds/Update~selected~feed},
varje flöde uppdateras även varje gång de markeras i listan.

Källkoden ligger i underkatalogen \verb!src!.

\section{Systembeskrivning}\label{Systembeskrivning}
% Beskriv översiktligt hur programmet är uppbyggt och hur det löser
% problemet.
Programmet är uppbyggt enligt \textit{Model-View-Controller}
modell. De tre huvudklasserna som driver programmet består av
\textit{JeedReader} som motsvarar programmets \textit{Controller}-del,
\textit{JeedView} som motsvarar \textit{View}-delen och
\textit{JeedModel} som motsvarar \textit{Model}-delen. Utöver dessa
finns även ett antal klasser som har hand om mer specifikt beteende,
främst är dessa klasser till hjälp för klassen \textit{JeedReader}.

\subsection{Feed.java}\label{Feed}
Den abstrakta klassen \textit{Feed} implementerar alla funktioner som
bör vara gemensamma för alla typer av flöden som ska kunna sparas i
programmet. Främst är ''Get''- och ''Set''-metoder implementerade för
information som rör flöden. Exempel på sådan information är flödets
titel, URL, typ, version, beskrivning.

Metoden \textit{toString()} är överlagrad för att returnera flödets
titel. Detta används i det grafiska gränssnittet för att skriva ut
flöden i en lista.

Källkod finns i bilaga \ref{Feed.java}.

\subsection{FeedItem.java}\label{FeedItem}
Den abstrakta klassen \textit{FeedItem} implementerade alla funktioner
som bör vara gemensamma för flödens artiklar. Främst är ''Get''- och
''Set''-metoder implementerade för information som rör
artiklar. Exempel på sådan information är artiklars titel,
beskrivning, om artikeln är läst eller ej.

Metoden \textit{toString()} är överlagrad för att returnera artikelns
status, det vill säga om den är läst eller ej och dess titel. En läst
artikel får strängen \textit{(Ny)} insatt före dess titel. Detta
används i det grafiska gränssnittet för att skriva ut artiklar i en
lista.

Källkod finns i bilaga \ref{FeedItem.java}.

\subsection{FeedOutputter.java}\label{FeedOutputter}
Klassen \textit{FeedOutputter} skriver ut flöden, \ref{Feed}, till
antingen en \textit{java.io.OutputStream} eller en
\textit{java.io.Writer}. I ett mellanled skapas tillfälligt en instans
av typen \textit{org.jdom.Document}, det är informationen i detta
objekt som sedan skrivs ut. I nuläget hämtas inte all information ur
flödet, se begränsningar sektion \ref{Begransningar}.

Källkod finns i bilaga \ref{FeedOutputter.java}.

\subsection{FeedParser.java}\label{FeedParser}
\textit{FeedParser} är ett gränssnitt som alla parsers i programmet
ska implementera. Gränssnittet innehåller en metod
\textit{parse(org.jdom.Document)} som returnerar ett flöde av typen
\textit{Feed}.

Källkod finns i bilaga \ref{FeedParser.java}.

\subsection{FeedUtil.java}\label{FeedUtil}
Klassen \textit{FeedUtil} innehåller enbart statiska
''hjälp\-metoder'' för resten av programmet. Metoder för att skapa
flöden från textsträngar eller av \textit{java.net.URL} är
implementerade.

Metoden \textit{getFeedFiles()}\label{getFeedFiles()} returnerar alla
flödes-filer som programmet sparar sina flöden i. Dessa filer ligger i
katalogen \textit{config}, bestämd av konstanten \textit{CONF\_DIR} i
klassen \textit{JeedReader}, och har filnamn som slutar med
\verb!feed.xml!.

Metoden \textit{getFeedParser(org.jdom.Document)} används internt i
klassen för att returnera en korrekt implementation av gränssnittet
\textit{FeedParser}. I nuläget returneras en instans av
\textit{RssParser}, se sektion \ref{RssParser}, om root-elementet i
det inskickade dokumentet är en \textit{rss}-tag och dess attribut
\textit{version} består av strängen ''2.0''. Om det inskickade
dokumentets root-element är en \textit{jss}-tag retureras en instans
av \textit{JeedConfigParser}, se sektion \ref{JeedConfigParser}.

Källkod finns i bilaga \ref{FeedUtil.java}.

\subsection{JeedConfigParser.java}\label{JeedConfigParser}
Klassen \textit{JeedConfigParser} utökar klassen \textit{RssParser}
och överlagrar dess metoder \textit{parse(org.jdom.Document)} och
\textit{parseItem(org.jdom.Element, RssItem)}. Denna utökade
funktionalitet av \textit{RssParser} hanterar dokument som är sparade
av själva programmet. Dessa dokument har en \textit{jss}-tag som root-element
artiklar med ett attribut \textit{isRead}. Metoden
\textit{parse(org.jdom.Document)} returnerar en instans av
\textit{RssFeed}, se sektion \ref{RssFeed}.

Källkod finns i bilaga \ref{JeedConfigParser.java}.

\subsection{JeedConfigWriter.java}\label{JeedConfigWriter}
Klassen \textit{JeedConfigWriter} använder sig av klassen
\textit{FeedOutputter}, för att spara en eller flera flöden som en
filer i katalogen som definieras av konstanten \textit{CONF\_DIR} i
klassen \textit{JeedReader}. Filnamnen som sparas bestäms av flödenas
\textit{URL}, och kommer därför alltid att bli unika för varje flöde.

Källkod finns i bilaga \ref{JeedConfigWriter.java}.

\subsection{JeedModel.java}\label{JeedModel}
Klassen \textit{JeedModel} är en representation av programmets
data. Här sparas alla flöden och metoder för att hämta, lägga till och
ta bort flöden är implementerade. Klassen utökar
\textit{java.util.Observable} och meddelar alla klasser som
registrerat sig som intressenter för denna klass när ändringar i
\textit{JeedModel}.
% TODO ändringar per flöde, skicka med flödet som argument

Källkod finns i bilaga \ref{JeedModel.java}.

\subsection{JeedReader.java}\label{JeedReader}
Klassen \textit{JeedReader} är programmets huvudklass, med denna klass
startas programmet. I \textit{Model-View-Controller}-modellen så
motsvarar denna klass \textit{Controller}, det vill säga klassen
ansvarar för att bygga upp programmet och hantera användares
handlingar.

\textit{JeedReader}s konstruktor skapar en \textit{JeedModel} och
skickar med denna instans till konstruktorn i \textit{JeedView}.

Flöden för alla filer, som fås av metoden
\textit{FeedUtil.getFeedFiles()}, se sektion \ref{getFeedFiles()},
skapas och läggs till i instansen av \textit{JeedModel}.

Inre klasser som implementerar olika typer av \textit{EventListener}s
deklareras och fästs vid olika komponenter i \textit{JeedView} genom
''set''-metoder i \textit{JeedView}.

Källkod finns i bilaga \ref{JeedReader.java}.

\subsection{JeedView.java}\label{JeedView}
Källkod finns i bilaga \ref{JeedView.java}.
\subsection{RssFeed.java}\label{RssFeed}
Källkod finns i bilaga \ref{RssFeed.java}.
\subsection{RssItem.java}\label{RssItem}
Källkod finns i bilaga \ref{RssItem.java}.
\subsection{RssParser.java}\label{RssParser}
Källkod finns i bilaga \ref{RssParser.java}.
\subsection{config}

\section{Begränsningar}\label{Begransningar}
% Vilka problem och begränsningar har din lösning av uppgiften? Hur
% skulle de kunna rättas till?

\section{Reflektioner}\label{Reflektioner}
% Var det något som var speciellt krångligt? Vilka problem uppstod och
% hur löste ni dem? Allmänna synpunkter. Hur skulle man kunna använda
% dessa metoder i andra mer omfattande system?

\section{Testkörningar}\label{Testkorningar}
% Noggranna testkörningar där man ser att programmet fungerar som det
% ska.

\section{Diskussion}\label{Diskussion}
% Hur fungerade det att följa en kodkonvention? Vilka var fördelarna
% respektive nackdelarna?


\newpage
\appendix
\pagenumbering{arabic}
\section{Källkod}\label{Kallkod}
% Källkoden ska finnas tillgänglig i er hemkatalog
% ~/edu/apjava/lab1/. Bifoga även utskriven källkod.
Härefter följer utskrifter från källkoden till denna laboration.

\subsection{Feed.java}\label{Feed.java}
\begin{footnotesize}
  \verbatiminput{../src/Feed.java}
\end{footnotesize}

\newpage
\subsection{FeedItem.java}\label{FeedItem.java}
\begin{footnotesize}
  \verbatiminput{../src/FeedItem.java}
\end{footnotesize}

\newpage
\subsection{FeedOutputter.java}\label{FeedOutputter.java}
\begin{footnotesize}
  \verbatiminput{../src/FeedOutputter.java}
\end{footnotesize}

\newpage
\subsection{FeedParser.java}\label{FeedParser.java}
\begin{footnotesize}
  \verbatiminput{../src/FeedParser.java}
\end{footnotesize}

\newpage
\subsection{FeedUtil.java}\label{FeedUtil.java}
\begin{footnotesize}
  \verbatiminput{../src/FeedUtil.java}
\end{footnotesize}

\newpage
\subsection{JeedConfigParser.java}\label{JeedConfigParser.java}
\begin{footnotesize}
  \verbatiminput{../src/JeedConfigParser.java}
\end{footnotesize}

\newpage
\subsection{JeedConfigWriter.java}\label{JeedConfigWriter.java}
\begin{footnotesize}
  \verbatiminput{../src/JeedConfigWriter.java}
\end{footnotesize}

\newpage
\subsection{JeedModel.java}\label{JeedModel.java}
\begin{footnotesize}
  \verbatiminput{../src/JeedModel.java}
\end{footnotesize}

\newpage
\subsection{JeedReader.java}\label{JeedReader.java}
\begin{footnotesize}
  \verbatiminput{../src/JeedReader.java}
\end{footnotesize}

\newpage
\subsection{JeedView.java}\label{JeedView.java}
\begin{footnotesize}
  \verbatiminput{../src/JeedView.java}
\end{footnotesize}

\newpage
\subsection{RssFeed.java}\label{RssFeed.java}
\begin{footnotesize}
  \verbatiminput{../src/RssFeed.java}
\end{footnotesize}

\newpage
\subsection{RssItem.java}\label{RssItem.java}
\begin{footnotesize}
  \verbatiminput{../src/RssItem.java}
\end{footnotesize}

\newpage
\subsection{RssParser.java}\label{RssParser.java}
\begin{footnotesize}
  \verbatiminput{../src/RssParser.java}
\end{footnotesize}
  
\end{document}