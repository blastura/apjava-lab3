\documentclass[a4paper, 12pt]{article}
\usepackage[swedish]{babel}
\usepackage[utf8]{inputenc}
\usepackage{verbatim}
\usepackage{fancyhdr}
\usepackage{graphicx}
\usepackage{parskip}
% Include pdf with multiple pages ex \includepdf[pages=-, nup=2x2]{filename.pdf}
\usepackage[final]{pdfpages}
% Place figures where they should be
\usepackage{float}

% Float for text
\floatstyle{ruled}
\newfloat{program}{thp}{lop}
\floatname{program}{XML}

% vars
\def\title{Web services}
\def\preTitle{Laboration 3}
\def\kurs{Applikationsprogrammering i Java, HT-08}

\def\namn{Anton Johansson}
\def\mail{dit06ajn@cs.umu.se}
\def\pathtocode{$\sim$dit06ajn/edu/apjava/lab3}

\def\handledareEtt{Johan Eliasson, johane@cs.umu.se}
\def\handledareTva{Tor Sterner-Johansson, tors@cs.umu.se}
\def\handledareTre{Daniel Henriksson, danielh@cs.umu.se}

\def\inst{datavetenskap}
\def\dokumentTyp{Laborationsrapport}

\begin{document}
\begin{titlepage}
  \thispagestyle{empty}
  \begin{small}
    \begin{tabular}{@{}p{\textwidth}@{}}
      UMEÅ UNIVERSITET \hfill \today \\
      Institutionen för \inst \\
      \dokumentTyp \\
    \end{tabular}
  \end{small}
  \vspace{10mm}
  \begin{center}
    \LARGE{\preTitle} \\
    \huge{\textbf{\kurs}} \\
    \vspace{10mm}
    \LARGE{\title} \\
    \vspace{15mm}
    \begin{large}
        \namn, \mail \\
        \texttt{\pathtocode}
    \end{large}
    \vfill
    \large{\textbf{Handledare}}\\
    \mbox{\large{\handledareTre}}
  \end{center}
\end{titlepage}

\newpage
\mbox{}
\vspace{70mm}
\begin{center}
% Dedication goes here
\end{center}
\thispagestyle{empty}
\newpage

\pagestyle{fancy}
\rhead{\today}
\lhead{\namn, \mail}
\chead{}
\lfoot{}
\cfoot{}
\rfoot{}

\cleardoublepage
\newpage
\tableofcontents
\cleardoublepage

\rfoot{\thepage}
\pagenumbering{arabic}

\section{Problemspecifikation}\label{Problemspecifikation}
% Beskriv med egna ord vad uppgiften gick ut på. Är det någonting som
% varit oklart och ni gjort egna tolkningar så beskriv dessa.

Problemspecifikation finns i original på:\\
\verb!http://www.cs.umu.se/kurser/5DV085/HT08/labbar/lab2.html!

\section{Användarhandledning}\label{Anvandarhandledning}
% Förklara var programmet och källkoden ligger samt hur man startar,
% kompilerar och använder det.
Programmet ligger i katalogen:\\
\texttt{\pathtocode}

Från denna katalog kompileras programmet med kommandot:
%TODO: flytta apache-tomcat rätt
\verb!salt:~/edu/apjava/lab3> ant!

De kompilerade filerna hamnar i underkatalogen \verb!bin!. Från denna
katalog körs programmet med kommandot

\verb!salt:~/edu/apjava/lab2/bin> java -cp .:../lib/jdom.jar JeedReader!

Alternativt kan programmet köras direkt från huvudmappen med
kommandot:

\verb!salt:~/edu/apjava/lab2> ant test-run!


Källkoden ligger i underkatalogen \verb!src!.

\section{Systembeskrivning}\label{Systembeskrivning}
% Beskriv översiktligt hur programmet är uppbyggt och hur det löser
% problemet.


\section{Begränsningar}\label{Begransningar}
% Vilka problem och begränsningar har din lösning av uppgiften? Hur
% skulle de kunna rättas till?

\section{Reflektioner}\label{Reflektioner}
% Var det något som var speciellt krångligt? Vilka problem uppstod och
% hur löste ni dem? Allmänna synpunkter. Hur skulle man kunna använda
% dessa metoder i andra mer omfattande system?

\section{Testkörningar}\label{Testkorningar}
% Noggranna testkörningar där man ser att programmet fungerar som det
% ska.

\section{Diskussion}\label{Diskussion}
% Hur fungerade det att följa en kodkonvention? Vilka var fördelarna
% respektive nackdelarna?

\newpage
\appendix
\pagenumbering{arabic}
\section{Källkod}\label{Kallkod}
% Källkoden ska finnas tillgänglig i er hemkatalog
% ~/edu/apjava/lab1/. Bifoga även utskriven källkod.
Härefter följer utskrifter från källkoden till denna laboration.

\newpage
\subsection{Entry.java}\label{Entry.java}
\begin{footnotesize}
  \verbatiminput{../src/se/umu/cs/edu/jap/highscoreservice/Entry.java}
\end{footnotesize}

\newpage
\subsection{HighScoreServiceClient.java}\label{HighScoreServiceClient.java}
\begin{footnotesize}
  \verbatiminput{../src/se/umu/cs/edu/jap/highscoreservice/HighScoreServiceClient.java}
\end{footnotesize}

\newpage
\subsection{RetrieveClient.java}\label{RetrieveClient.java}
\begin{footnotesize}
  \verbatiminput{../src/se/umu/cs/edu/jap/highscoreservice/RetrieveClient.java}
\end{footnotesize}

\newpage
\subsection{StoreClient.java}\label{StoreClient.java}
\begin{footnotesize}
  \verbatiminput{../src/se/umu/cs/edu/jap/highscoreservice/StoreClient.java}
\end{footnotesize}

\newpage
\subsection{FailureFaultException.java}\label{FailureFaultException.java}
\begin{footnotesize}
  \verbatiminput{../src/se/umu/cs/edu/jap/highscoreservice/stubs/FailureFaultException.java}
\end{footnotesize}

\newpage
\subsection{HighScoreServiceData.java}\label{HighScoreServiceData.java}
\begin{footnotesize}
  \verbatiminput{../src/se/umu/cs/edu/jap/highscoreservice/stubs/HighScoreServiceData.java}
\end{footnotesize}

\newpage
\subsection{HighScoreServiceMessageReceiverInOut.java}\label{HighScoreServiceMessageReceiverInOut.java}
\begin{footnotesize}
  \verbatiminput{../src/se/umu/cs/edu/jap/highscoreservice/stubs/HighScoreServiceMessageReceiverInOut.java}
\end{footnotesize}

\newpage
\subsection{HighScoreServiceSkeleton.java}\label{HighScoreServiceSkeleton.java}
\begin{footnotesize}
  \verbatiminput{../src/se/umu/cs/edu/jap/highscoreservice/stubs/HighScoreServiceSkeleton.java}
\end{footnotesize}

\newpage
\subsection{XMLUtil.java}\label{XMLUtil.java}
\begin{footnotesize}
  \verbatiminput{../src/se/umu/cs/edu/jap/highscoreservice/util/XMLUtil.java}
\end{footnotesize}

\newpage
\subsection{store-linux.sh}\label{store-linux.sh}
\begin{footnotesize}
  \verbatiminput{../store-linux.sh}
\end{footnotesize}

\newpage
\subsection{retrieve-linux.sh}\label{etrieve-linux.sh}
\begin{footnotesize}
  \verbatiminput{../retrieve-linux.sh}
\end{footnotesize}

\newpage
\subsection{highscoreservice.wsdl}\label{highscoreservice.wsdl}
\begin{footnotesize}
  \verbatiminput{../wsdl/highscoreservice.wsdl}
\end{footnotesize}

\end{document}